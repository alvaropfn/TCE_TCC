% resumo na língua vernácula (obrigatório)
\setlength{\absparsep}{18pt} % ajusta o espaçamento dos parágrafos do resumo
\begin{resumo}

	É competência do Tribunal de Contas do Estado do Rio Grande do Norte (TCE-RN) o controle externo sobre a administração pública de acordo com o que rege a Constituição Federal de 1988, assim como a prestação de contas à sociedade das entidades jurisdicionadas sob sua guarda. Para a execução dessas ações o TCE-RN desenvolve e utiliza diversos Sistemas Web, seja para permitir que entidades jurisdicionadas cadastrem ou acesse informações referentes ao escopo de interesses do TCE-RN ou permitir a divulgação de dados a população e aos Jurisdicionados como determinada pela Lei Complementar 131 também conhecida como Lei de Transparência. Diversos processos e tecnologias são usados no desenvolvimento dos sistemas do TCE-RN e o objetivo deste trabalho será propor um sistema de design para prototipação de interfaces de usuário que seja ágil, colaborativo, versionável e adaptável, criando um conjunto componentes para web.
	% \lipsum[1-1]

	\noindent
	\textbf{Palavras-chaves}: Sistemas de design. Prototipação. Biblioteca de componentes. Figma.
\end{resumo}
% ---
% resumo em inglês
\begin{resumo}[Abstract]
	\begin{otherlanguage*}{english}

		The Court of Auditors of the State of Rio Grande do Norte (TCE-RN) is responsible for external control over public administration in accordance with the 1988 Federal Constitution, as well as the rendering of accounts to society of the jurisdicted entities under its guard. For the execution of these actions, TCE-RN develops and uses several Web Systems, either to allow jurisdicted entities to register or access information related to the scope of interests of TCE-RN or to allow the disclosure of data to the population and to jurisdicted's as determined by Law Complementary 131 also known as the Transparency Law. Several processes and technologies are used in the development of TCE-RN systems and the objective of this work will be to propose a user interface design system for prototyping that is agile, collaborative, versionable and adaptable, creating a set of components for the web.
		% 	\lipsum[1-1]

		\vspace{\onelineskip}
		\noindent
		\textbf{Keywords}: Design System. Prototyping. tComponent Library. Figma.
	\end{otherlanguage*}
\end{resumo}