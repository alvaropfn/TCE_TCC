\chapter[Capítulo 2]{2 Estado da Arte e Fundamentação Teórica}
\label{ch:cap2}

  Neste capítulo será apresentado o estado da arte referente a construção de sistemas de designs e prototipação assim como o material utilizado como embasamento teórico para o seu desenvolvimento.

\section[\textit{Seção}]{\textit{Afinal o que é um sistema de design}}
  Realizando uma breve pesquisa na internet a expressão mais comumente encontrada para descrever o que seria um sistema de design remetem a Nathan Curtis e a frase \textit{"A Design System isn’t a Project. It’s a Product, Serving Products".} \cite{descricao_sistema_design} Poderiamos traduzir para Um sistema de design não é um projeto. É um produto servindo outros produtos.

  Um sistema de design é a representação viva de um conjunto de documentos com o intuito de unificar a identidade visual de um ecosistema de produtos de uma instituição. Para servir a este proposito esses documentos não podem ser estáticos, devem permitir constantes mudanças e conseguir manter a ideia central de poder encaixar formas sem perder consistência \cite{design_gov_digital}.
  Em 2019 o Governo Federal decidiu iniciar a construção do seu sistema  de design com uma motivação clara

\begin{citacao}[brazil]
  [...]A iniciativa potencializa a eficiência e a eficácia dos usuários na utilização de interfaces para acesso aos serviços e aos sistemas de Governo, possibilitando uma única curva de aprendizado e garantindo a previsibilidade na utilização dos diferentes sistemas. \cite{design_gov_federal}
\end{citacao}

  Para Gonzalez um sistema de design é "[...]um conjunto de padrões para design e código, componentizado que unificam as duas práticas". \cite{GuilhermeGonzalez} Para o Governo Federal um \textit{Design System} "[...]tem como objetivo guiar todos os responsáveis pela construção de interfaces interativas orientadas à experiência única do usuário, considerando a acessibilidade e a usabilidade dos sistemas". \cite{design_gov_federal}

  É fato que grandes entidades vem adotando a elaboração de sistemas de design com o intuito de se beneficiar das vantagens proporcionadas pela unificação da lingagem unificada de comunicação com usuarios de seus sitemas, acelerar o processo de prototipação de seus layoutes e interfaces e, deminuir o tempo de desenvolvimento das aplicações por meio do respaldo de bibliotecas de componentes mantidas por esses sistemas.

  Para citar algumas das empresas e entiades que possuem ou estão e precesso de desenvolvimento de um sitema de design:

\begin{itemize}
  \item \href{https://material.io/design/}{Google}
  \item \href{https://www.microsoft.com/design/fluent/#/}{Microsoft}
  \item \href{https://developer.apple.com/design/}{Apple}
  \item \href{https://www.carbondesignsystem.com/}{IBM}
  \item \href{https://www.figma.com/community/plugin/860845891704482356/GitLab}{GitLab}
  \item \href{https://www.gov.br/governodigital/pt-br/transformacao-digital/ferramentas/design-system}{Governo Federal}
  \item \href{https://www.gov.uk/guidance/government-design-principles}{Governo do Reino Unido }
  \item \href{https://designsystem.gov.au/}{Governo da Australia}
\end{itemize}

\section[\textit{Seção}]{\textit{Principios de um sistema de design}}

  Existem uma quantidade virtualmente infinita de maneiras de se construir um \textit{Design System}. Contudo uma das maneiras mais comuns de se proceder com seu desenvolvimento consiste em separar o conjunto de documentos gerados, agrupando-os em categorias de acordo com a sua finalidade.

  No caso do Governo fedaral por exmeplo temos: Fundamentos Visuais, Componentes, Templates, Guias e Orientações. o Gitlab por sua vez organizou seu sistema em: Templates e Recursos

  % TODO citar tecnologias usadas no desenvolvimento de sistmeas de design