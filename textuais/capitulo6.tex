\chapter[Capítulo 6]{Considerações finais}
\label{ch:cap6}

Neste capítulo trago minhas considerações finais sobre o que foi alcançado e propostas para o amadurecimento do que foi desenvolvido.

\section{Conclusões}\label{cap6:conclusao}

É fato que um sistema de design é um documento vivo e em evolução permanentemente e com este trabalho foi possível realizar o desenvolvolvimento de um SD com a sobriedade, a objetividade e o senso de propósito pertinente ao Tribunal de Contas do Estado do Rio Grande do Norte, sendo assim, é considerado pelo autor que as intenções e objetivos do trabalho foram alcançados com êxito, evidentemente não sem sacríficios em relação as escolhas adotadas, pois o desenvolvimento de um sistema de design não é algo que se conclui, mas que se abandona, pois o fim dos prazos chegam e o produto precisa ser entregue de um jeito ou de outro, mas sempre passível de melhorias, críticas e reinterpretações. Neste trabalho não foi diferente e apesar do desejo de continuar a evolução do que vinha-se desenvolvendo a entrega tinha que ser efetuada.

\section{Melhorias}\label{cap6:melhorias}

Dentro das muitas melhorias possíveis, planejadas ou simplesmente encontradas podemos destacar, acima de tudo, a falta de comunicação direta com a biblioteca de componentes de front-end tce-ng-lib. Era esperado que as alterações propostas fossem incluídas nos componentes de front-end e a nomenclatura e descrição das duas poderiam ser melhoradas para entrar em conformidade, tornando a busca e compreensão facilitadas tanto para os designers quanto para os desenvolvedores, refernciando os componentes e as classes de estilo pelos memos nomes tanto no angular quanto no figma. Outro grande problema foi a ausência de um conjunto de componentes para dispositivos móveis, que foi pensado em se fazer inicialmente mas o estudo e preparatório e as sucessivas revisões do trabalho consumiram muito tempo e o receio de não conseguir terminar os documentos a tempo inibiram o início desta fase do trabalho.

% \section{Melhorias}\label{cap5:melhorias}