\chapter[Introdução]{Introdução}
\label{ch:introducao}

  O Tribunal de Contas do Estado do Rio Grande do Norte (TCE-RN) segundo \cite{Legislativa} é o orgão responsavel pelo controle externo do Poder Legislativo Municipal, cabendo a este o parecer à prestação de contas que as prefeituras ou qualquer outro orgão sob sua jurisdição deve prestar anualmente.

  A exceção dos dados sob sigilo ou restrições da lei nº 12.557 \cite{lei_12527} é dever do Poder Executivo Federal promover a cultura de transparência pública a publicação de dados contidos em bases de dados de orgãos e entidades da administração pública federal direta, autárquica e fundacional sob a forma de dados abertos assim como fomentar o controle social e o desenvolvimento de novas tecnologias destinadas à construção de ambiente de gestão pública participativa e democrática e à melhor oferta de serviços públicos para o cidadão \cite{dec_8777}.

  Embora seja um tribunal o Tribunal de Contas não encontra-se circunscrito nem faz parte do Poder Judiciário, pois seu caráter é de naturazea, administrativa (contábil), trabalhando em parceria e nao em subordinação ao  ao Judiciário \cite{barreto_tribunais}.
  
  Portanto como corte cuja finalidade é a fiscalização de Prefeituras dos Municípios o TCE-RN 

\section{Uma subseção explicativa}

Lorem ipsum, uma citação direta 



Para compreender melhor as grandes mudanças e os benefícios gerados pelas \textit{Smart Grids} no contexto do fornecimento elétrico, a \autoref{tab-comparativa} traz um breve comparativo entre as redes tradicionais e as redes inteligentes.



\section{Trabalhos Relacionados}
\lipsum[1-1]

\section{Motivação}
O que lhe motiva a realizar este trabalho.

\section{Objetivos}
Objetivo geral e específicos.

\section{Estrutura do Trabalho}
Este trabalho apresenta uma introdução sobre o tema, mostrando os fatores que motivam a implantação da ideia, além da justificativa e dos objetivos. Em sequência, o \autoref{ch:cap2} aborda (...). O \autoref{ch:cap3}, por sua vez, explica a metodologia para ..., enquanto o \autoref{ch:cap4} trata de (...). O \autoref{ch:cap5} apresenta (...). Por fim, o \autoref{ch:cap6} traz as principais conclusões e contribuições deste trabalho.
