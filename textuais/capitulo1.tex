\chapter[Introdução]{Introdução}
\label{ch:introducao}

  O Tribunal de Contas do Estado do Rio Grande do Norte (TCE-RN) segundo \cite{Legislativa} é o orgão responsavel pelo controle externo do Poder Legislativo Municipal, cabendo a este o parecer à prestação de contas que as prefeituras ou qualquer outro orgão sob sua jurisdição deve prestar anualmente.

  A exceção dos dados sob sigilo ou restrições da lei nº 12.557 \cite{lei_12527} é dever do Poder Executivo Federal promover a cultura de transparência pública a publicação de dados contidos em bases de dados de orgãos e entidades da administração pública federal direta, autárquica e fundacional sob a forma de dados abertos assim como fomentar o controle social e o desenvolvimento de novas tecnologias destinadas à construção de ambiente de gestão pública participativa e democrática e à melhor oferta de serviços públicos para o cidadão \cite{dec_8777}.

  Embora seja um tribunal o TCE-RN não encontra-se circunscrito nem faz parte do Poder Judiciário, pois seu caráter é de naturazea, administrativa (contábil), trabalhando em parceria e nao em subordinação ao  ao Judiciário \cite{barreto_tribunais}.

  Portanto como corte cuja finalidade é a fiscalização de Prefeituras dos Municípios e unidades jurisdicionados o TCE-RN precisa de transparência e corretude dos dados a serem avaliados afim de fornecer essa mesma transparência e corretude de volta aos cidadãos, aqueles a quem realmente servem como orgão público.

  Para atender a crescente demanda da sociedade pela ampliação da transparência O TCE-RN oferece o acesso a uma Interface de Programação de Aplicativos (API) ,do inglês \textit{Application Programming Interface}, no formato de Dados Abertos sem restrições de uso possibilitando a criação de aplicações digitais \cite{dados_abertos_tcern} que façam uso dos dados do seu Sistema Integrado de Auditoria Informatizada (SIAI) permitindo que não só o cidadão desenvolva aplicativos que formentem o controle social, como as equipes de desenvolvimento do tribunal, facilitando o compartilhamento de dados entre aplicações internas ao tribunal desenvolvidas pela Diretoria de Informática do TCE-RN (DIN).

  A DIN é a unidade responsável por desenvolver, gerenciar e fornecer serviços de tecnologia da informação para o Tribunal \cite{relatorio_trimestral}. ate o primeiro trimestre de 2019 a DIN executou com a ajuda do seu corpo de servidores, estagiarios, do Instituto Metrópole Digital (IMD), e da Indra, empresa terceirizada que presta serviços de Sofware house ao TCE-RN, ao menos 12 projetos e 3718 atendimentos técnicos ao Tribunal.

\section{Apresentação do problema }

  A DIN conta entre servidores, estagiários, terceirizados, e outros colaboradores com um total de 58 membros na sua unidade \cite{relatorio_trimestral} que entre outras demandas precisam desenvolver sistemas e ferramentas para melhor atender não só as necessidades internas da sua diretoria como a de entidades externas jurisdicionadas seja para consumir ou fornecer dados ao Tribunal de Contas do Estado.

  Uma das grandes dificuldades atualmente no tribunal está em manter a coesão entre tantos sistemas criados ao longo da sua historia. Outro problema está na velocidade de produção desses sistemas que variam imensamente em escopo e complexidade, passando por longos periodos de levantamento de requisitos, prototipação e produção até a sua homologação. Além disso apesar de fornecer acesso a informação por meio de APIs, caso o cidadão deseje desenvolver um sistema ou serviço baseado nos dados abertos fornecidos pelo tribunal, os mesmos não podem contar com outra referência aos modelos de interface humano computador além das telas de sistemas com acesso público providas pelo TCE.

% TODO: achar a referncia da quantidade de projetos
  % Apesar de parecer extenso o corpo de membros da Diretoria de Informática pode ser considerado pequeno pois, lida com uma grande quantidade de sistemas já existentes (68 em produção no terceiro trimestre de 2020). Esses sistemas precisam constantemente passar por fases de manutenção e evolução enquanto novos sistemas vem sendo de senvolvidos com o passar do tempo.

  % O problema que este trabalho visa enfrentar compreende parte do processo de desenvolvimento e validação de novos sistemas ou funcionalidades para sistemas preexistentes do TCE-RN por meio da construção de prototipos de alta fidelidade para validação junto aos clientes dos produtos desenvolvidos pelo tribunal usando um sitema unico de design que mitigue as inconsistências visuais, de interação e interpretação dos usuarios. Um sistema que possa ser colaborativamente editado, evoluido e versionado alem de perimtir a sua utilização por parte da população que deseje consumir e trabalhar com dados fornecidos pelo TCE-RN e manter a consistência visual do tribunal.

\section{Apresentação da solução }

  Para mitigar os problemas anteriormente citados de acesso publico aos modelos de interfaces e favorecer a coesão entre os sistemas novos a serem desenvolvidos a proposta deste trabalho é o desenvolvimento de um modelo de identidade visual composto por um sistema de design e uma biblioteca de componentes reutilizáveis que seja colaborativamente editável, evoluido e versionado. Permitindo a rápida prototipação de interfaces de usuários de alta fidelidade que sejam facilmente, testaveis pelos clientes da DIN e que permitam ao cidadão criar sistemas ou serviços com uma experiência de usuário única em torno do ecosistema do tribunal.


% TODO: consturir um proble melhor
%* tentar ligar a nova necessidade do TCE em criar aplicacoes para dispositivos moveis as vantagens da prototipacao rapida e de alta f
  % Dentro das diversas etapas do Tribunal para o desenvolvimento de novos sistemas podemos dentre os multiplos documentos gerados destacar a construção dos prototipos de tela, que atualmente usa o Sofware Open-Source Pencil Project. Contudo o Pencil é uma ferramenta de antiga (2008) que apesar de contuar com funcionalidades como coleções de formas pre definidas e suporte a multiplas plataformas carece das vantagens de integração e agilidade proporcionadas por plataformas basedas na web.

  \section{Estrutura do Trabalho}
  Este trabalho apresenta inicialmente uma introdução sobre as obrigações e motivações do Tribunal de Contas do Estado do Rio Grande do Norte e dos fatores que levanta as justificativas da elaboração deste trabalho.
  Na sequêcia o \autoref{ch:cap2} contendo o embasamento teórico e o estado da arte para a elaboração deste trabalho, em seguida o \autoref{ch:cap3} introduz modelos e exemplos dos documentos gerados pela ferramenta desenvolvida assim como a descrição do trabalho produzido. O \autoref{ch:cap4} trata dos resultados obtidos com o que foi desenvolvido enquanto no \autoref{ch:cap5} são abordadas as conclusões, possivéis contribuições e melhorias propostas.
