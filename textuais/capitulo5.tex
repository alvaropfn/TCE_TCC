\chapter[Capítulo 5]{Conclusões e Propostas de Melhorias}
\label{ch:cap5}

  Neste capítulo trago minhas considerações finais sobre o qe foi alcançado e propostas para o amadurecimento do que foi desenvolvido.

\section{Conclusões}\label{cap5:conclusao}

  É fato que um sistema de design é um documento vivo e em evolução permanentemente e com este trabalho foi possivel realizar o desenvolvolvimento de um SD com a sobriedade, a objetividade e o senso de proposito pertinente ao Tribunal de Contas do Estado do Rio Grande do Norte, sendo assim é considerado pelo autor que as intenções e objetivos do trabalho foram alcançados com exito, evidentemente não sem sacrificios em relação as escolhas adotadas pois o desenvolvimento de um sistema de design não é algo que se conclui, mas que se abandona, pois o fim dos prazos chegam e o produto precisa ser entregue de um geitou ou de outro mas sempre passivel de melhorias, criticas e reenterpretações. Neste trabalho não foi diferente e apesar do desejo de continuar a evolução do que vinha-se desenvolvendo a entrega tinha que ser efetuada.

\section{Melhorias}\label{cap5:melhorias}

  Dentro das muitas melhorias possiveis, planejadas ou simplesmente encontradas podemos destacar acima de tudo a falta de comunicação direta com a biblioteca de componentes de front-end tce-ng-lib. Era esperado que as alterações propostas fossem incluidas nos componentes de front-end e a nomenclatura e descrição das duas poderiam ser melhoradas para entrar em conformidade tornando a busca e compreensão facilitadas. Outro grande problema foi a ausencia de um conjunto de componentes para dispositivos moveis, que foi pensado em se fazer inicialmente mas o estudo e preparatorio e as sucessivas revisões do trabalho consumiram muito tempo e o medo de nao conseguir terminar os documentos a tempo inibiram o inicio desta fase do trabalho.

  % \section{Melhorias}\label{cap5:melhorias}